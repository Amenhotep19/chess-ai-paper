% !TeX root = ../paper.tex
\section*{Abstract in deutscher Sprache}

Die vorliegende Arbeit stellt die Implementierung eines Schachcomputers dar.
Hierfür werden zuerst die theoretischen Grundlagen, die für die Erstellung des Schachcomputers notwendig sind, erläutert.
Dies umfasst den Mini-Max-Algorithmus, die Bewertungsfunktion und die in der Arbeit verwendete Bewertungsheuristik sowie die Alpha-Beta-Suche.
Des Weiteren werden als Verbesserungen der Alpha-Beta-Suche die Zugvorsortierung, die Ruhesuche und Transpositionstabellen erklärt.
Es folgt die Implementierung des Schachcomputer mittels der Programmiersprache Python.
Den Abschluss der Arbeit bildet ein Auswertungs- und Diskussionsteil.

\acresetall{}

\newpage \section*{Abstract in englischer Sprache}

The paper at hand shows the implementation of a chess computer.
Therefore, theoretical basics such as the Mini-Max-Algorithm, the used evaluation heuristic as well as Alpha-Beta-Pruning are explained.
Furthermore, enhancements of Alpha-Beta-Pruning namely move ordering, quiescence search, and transpositional tables are described.
Subsequently, the actual implementation of the chess computer is shown followed by an evaluation and discussion section.

\acresetall{}
