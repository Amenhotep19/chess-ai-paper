% !TeX root = ../../paper.tex
\subsection{Alpha-Beta Pruning}

Beim \textit{Alpha-Beta Pruning} (zu dt. etwa \textit{Alpha-Beta Beschneidung}) handelt es sich um ein Verfahren, das zum Ziel hat, den exponentiell wachsenden Spielbaum mit seiner rasch anwachsenden Zahl an Knoten zu reduzieren.
Diese Reduktion ist deshalb sinnvoll, da rund 99\% der Knoten des Spielbaums im Schach in der aktuellen Situation uninteressant sind.
Die Reduktion führt zu einem kleineren Spielbaum, wodurch der Schachcomputer einen wesentlichen Leistungsschub erhalten kann [\cite{Paulsen2009}].
Im Vergleich zum Mini-Max Verfahren werden so für den Spieler als nicht vorteilhafte Spielzüge erscheinende Möglichkeiten nicht weiterverfolgt, wodurch das Alpha-Beta Pruning als eine optimierte Variante des Mini-Max Verfahrens angesehen werden kann [\cite{Knuth1975}].
Auch ist dieses Verfahren an das Verhalten menschlicher Spieler angelegt.
So blenden menschliche Spieler Spielzüge aus bzw. verfolgen diese nicht weiter, wenn sie ihnen unsinnig erscheinen [\cite{Paulsen2009}].


\subsubsection{Ruhesuche}

