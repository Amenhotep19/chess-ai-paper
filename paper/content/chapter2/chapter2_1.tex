% !TeX root = ../../paper.tex
\subsection{Bewertungsfunktion}

In der Spieltheorie ist es in der Regel nicht möglich, alle möglichen Zugfolgen aus einer Spielposition heraus bis zum Ende zu verfolgen.
Deshalb \glqq [...] wird eine Funktion benötigt, die die Stellung auf dem Spielbrett danach bewertet,
ob sie für eine der beiden Parteien vorteilhaft oder nachteilig ist.\grqq \ [\cite{Paulsen2009}]
Diese Funktion wird als \textit{Bewertungsfunktion} bezeichnet.
Die Bewertungsfunktion setzt sich aus einem materiellen und einer positionellen Komponente zusammen.
Bei der materiellen Komponente werden die verbleibenden Figuren auf der eigenen Seite gezählt und die Anzahl der gegnerischen Figuren von diesem Wert subtrahiert.
Daraus lassen sich erste Schlussfolgerungen über den Spielstand ziehen.
Zudem ist es möglich, die derzeitige Spielphase abzuleiten [\cite{Paulsen2009}].

Da es beim Schach aber auch entscheidend ist, auf welchen Positionen sich die einzelnen Figuren befinden und in welcher Position sie zueinander stehen (Bauernstruktur, Königssicherheit), wird zusätzlich zur materiellen Komponente eine positionelle Komponente berechnet.
Die verschiedenen Positionen werden aus der Spielbrettaufstellung entnommen und fließend mit ihren jeweiligen Gewichtungen in die anschließende Bewertung ein [\cite{Paulsen2009}].


\subsubsection{Einfache Bewertungsfunktion}

Das Schachspiel setzt sich aus zwei wichtigen Faktoren zusammen: Einerseits ist abgesehen von der Tatsache, welcher Spieler den ersten Zug macht, keine Zufallskomponete enthalten, andererseits handelt es sich um ein Spiel mit \gls{PI}.
Diese zwei Faktoren führen dazu, dass bei jeder Schachbrettposition eine der folgenden drei Aussagen gilt [\cite{Shannon1950}]:

\begin{enumerate}
    \item Es handelt sich um eine gewonnene Position für Weiß. Somit kann Weiß einen Sieg forcieren, wobei Schwarz verteidigen.
    \item Es handelt sich um eine gewonnene Position für Schwarz. Somit kann Schwarz einen Sieg forcieren, wobei Weiß spielt.
    \item Es handelt sich um eine unentschiedene Position für beide Parteien. Somit kann es nur ein Unentschieden am Ende geben, falls beide Parteien dies forcieren und keine Fehler machen.
\end{enumerate}

\noindent Bei einigen Spielen (so auch beim Schach) lässt sich aus den genannten zwei Faktoren und den daraus resultierenden drei Aussagen eine \textit{einfache Bewertungsfunktion} \(\displaystyle f(P)\) ableiten, wobei \(\displaystyle P\) die Schachbrettposition bezeichnet.
Der Rückgabewert der Funktion ist die Kategorie, in die die jeweilige Position gehört: Sieg (+1), Unentschieden (0), Niederlage (-1).
Zum Zeitpunkt des Zuges des Schachcomputers werden die Werte \(\displaystyle f(P)\) für alle Positionen nach möglichen Halbzügen berechnet.
Der Zug mit dem maximalen Wert wird am Ende ausgeführt [\cite{Shannon1950}].

Im Zuge dieser Arbeit wird die einfache Bewertungsfunktion nach Tomasz Michniewski verwendet, die ursprünglich in der \textit{Polish chess programming discussion list (progszach)} veröffentlicht wurde und im chessprogramming.org Wiki beschrieben wird.
Die Gewichtungen der Bewertungsfunktion wurden so gewählt, dass ein Mangel an Schachkenntnissen ausgeglichen werden kann und nicht, um durch Schachkenntnisse ergänzt zu werden.
Da die Gewichtungen von Tomasz Michniewski und nicht von den Autoren dieser Arbeit festgelegt wurden, erfolgt an dieser Stelle keine Erläuterung zur Entstehung der Gewichtungswerte [\cite{Wiki2018}]. Die von Tomasz Michniewski beschriebene einfache Bewertungsfunktion wird in zwei Teilen dargestellt: Einfache Figurenwerte (engl. \textit{simple piece values}) und Figuren-Quadrat Tabellen (engl. \textit{piece-square tables}).

Mit der Festlegung von Werten je Figur werden vier verschiedene Ziele erreicht:

\begin{enumerate}
    \item Vermeidung des Austauschs einer kleinen Figur gegen drei Bauern
    \item Dem Computer signalisieren, dass das halten des Läuferpaars vorteilhaft ist
    \item Vermeidung des Austauschs von zwei kleinen Figuren gegen einen Turm und einen Bauern
    \item Verbleiben bei der menschlichen Schacherfahrung
\end{enumerate}

\noindent Der erste Punkt wird durch Gleichung~\ref{eq:piece-values_first-result} erfüllt.

\begin{equation} \label{eq:piece-values_first-result}
\begin{split}
    L > 3B \\
    S > 3B
\end{split}
\end{equation}

\noindent Zwar gibt es durchaus Positionen, in denen drei Bauern wertvoller als eine kleine Figur sind, jedoch ist es im Allgemeinen besser eine kleine Figur zu behalten, da im Spielverlauf die Bauern individuell attackiert werden können und deren Wert als Dreier-Figuren-Gespann verloren geht [\cite{Wiki2018}].
Der zweite genannte Punkt wird durch Gleichung~\ref{eq:piece-values_second-result} erreicht.

\begin{equation} \label{eq:piece-values_second-result}
    L > S
\end{equation}

\noindent Zwar garantiert diese Gleichung kein Halten des Läuferpaars, da am Ende ein Läufer gegen einen Springer stehen kann, dennoch ist es eine Tatsache, dass Spieler oftmals Springer mit Läufern schlagen und nicht Läufer mit Springern [\cite{Wiki2018}].
Die ersten beiden Gleichungen~\ref{eq:piece-values_first-result} und~\ref{eq:piece-values_second-result} führen zusammen zu Gleichung~\ref{eq:piece-values_first-and-second-result-combined}.

\begin{equation} \label{eq:piece-values_first-and-second-result-combined}
    L > S > 3B
\end{equation}

\noindent Der dritte Punkt wird zwar durch Gleichung~\ref{eq:piece-values_third-result} erreicht, dennoch haben Spiele wie jenes zwischen Karpov und Kasparov gezeigt, dass bereits ein Turm und zwei Bauern ausreichen, um gegen zwei kleine Figuren gewinnen zu können.

\begin{equation} \label{eq:piece-values_third-result}
    L + S > T + B
\end{equation}

\noindent Aus diesem Grund wird die Gleichung~\ref{eq:piece-values_third-result}, die ferner als Gleichung~\ref{eq:piece-values_third-result-full} dargestellt werden kann, um einen Faktor erweitert, woraus sich Gleichung~\ref{eq:piece-values_third-result-enhanced} ergibt [\cite{Wiki2018}].

\begin{equation} \label{eq:piece-values_third-result-full}
    T + 2B > L + S > T + B
\end{equation}

\begin{equation} \label{eq:piece-values_third-result-enhanced}
    L + S > T + 1.5B
\end{equation}

\noindent Durch die hier beschriebenen Gleichungen ist der vierte Punkt implizit erfüllt.
Zu guter Letzt wird noch eine Gleichung benötigt, die die Überlegenheit einer Dame-Bauer-Kombination gegenüber zwei Türmen darstellt.
Dies ist in Gleichung~\ref{eq:piece-values_final-result-missing} abgebildet.

\begin{equation} \label{eq:piece-values_final-result-missing}
    Q + B = 2T
\end{equation}

\noindent Somit erhält man das Gleichungssystem, das in Gleichung~\ref{eq:piece-values_final-result} dargestellt ist.

\begin{equation} \label{eq:piece-values_final-result}
\begin{split}
    L > S > 3B \\
    L + S = T + 1.5B \\
    Q + B = 2T
\end{split}
\end{equation}

\noindent Dieses Gleichungssystem wird durch die in Tabelle~\ref{tb:piece-values_satified} gelisteten Werte erfüllt.
Diese Werte wurden von Tomasz Michniewski festgelegt mit Ausnahme des Wertes des Königs.
Dieser Wert stammt aus dem Paper von Claude Shannon, wobei dieser an der Stelle dem König den Wert 200 gibt, was durch die Umrechnung in Hundertstelbauer zu 20000 führt [\cite{Shannon1950}].
Der Wert für den König ist bewusst so hoch gewählt worden, da der Verlust des Königs automatisch zur Niederlage führt.
Daraus folgt zudem, dass der Verlust des Königs durch die materielle Komponente der Bewertungsfunktion einfacher erkannt werden kann.
Zu guter Letzt sei angemerkt, dass die Wahl der Figurenwerte dazu führt, dass diese in einer 2~Byte vorzeichenbehafteten Ganzzahl abgelegt werden können, da der Gesamtwert aller Spielfiguren einer Seite rund 30300 beträgt [\cite{Wiki2018}].

\begin{table}
    \centering
    \begin{tabular}{| L{0.47\textwidth} | R{0.47\textwidth} |}
        \hline
        \thead{Figur} & \thead{Wert} \\
        \hline
        Bauer & 100 \\
        \hline
        Springer & 320 \\
        \hline
        Läufer & 330 \\
        \hline
        Turm & 500 \\
        \hline
        Dame & 900 \\
        \hline
        König & 20000 \\
        \hline
    \end{tabular}
    \caption[Figurenwerte in Hundertstelbauer]{Figurenwerte in Hundertstelbauer [\cite{Wiki2018}]}
    \label{tb:piece-values_satified}
\end{table}


\subsubsection{Weitere Bewertungsfunktionen}

